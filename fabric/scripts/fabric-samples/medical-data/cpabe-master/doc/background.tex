%%%%%%%%%%%%%%%%%%%%%%%%%%%%%%%%%%%%%%%%%%%%%%%%%%%%%%
% Filename:      background.tex
% Author:        Junwei Wang(wakemecn@gmail.com)
% Last Modified: 2012-04-21 22:01
% Description:
%%%%%%%%%%%%%%%%%%%%%%%%%%%%%%%%%%%%%%%%%%%%%%%%%%%%%%
\chapter{背景}
我们首先给出CP-ABE安全性的正式定义,然后我们给出双线性映射的背景信息。如同Goyal等人
\cite{GPSW:ABE}所做,我们定义访问结构并在我们的安全性定义中使用;不同的是,在我们的
定义中属性将用来描述用户,访问结构用来标记被加密的数据。
%%%%%%%%%%%%%%%%%%%%%%%%%%%%%%%%%%%%%%%%%%%%%%%%%%%%%%%
%%%%%%%%%%%%%%%%%%%%%%%%%%%%%%%%%%%%%%%%%%%%%%%%%%%%%%%
\section{定义}
\begin{Definition}[访问结构\cite{Beimel:SSSS}]
令$\{P_1,P_2,...,P_n\}$是所有的参与方集合。集合$\mathbb{A}\subseteq 2^{\{P_1,P_2,...,P_n\}}$
是单调的是指:对于任意$B,C$,如果$B\in\mathbb{A}$且$B\subseteq C$,则$C\in\mathbb{A}$。
访问结构(特别的,单调访问结构)是$\{P_1,P_2,...,P_n\}$的非空子集,例如,
$\mathbb{A}\subseteq 2^{\{P_1,P_2,...,P_n\}}\setminus\{\emptyset\}$,
的集合$\mathbb{A}$(特别的,单调集合)。集合$\mathbb{A}$中的集合被称为授权集,不在
集合$\mathbb{A}$中的集合被称为未授权集。
\end{Definition}
\par
在我们的论述中,参与方被属性所代替,因此,访问结构$\mathbb{A}$包含了授权属性集。本文中
我们只注意单调访问结构,但是,以我们的技术实现通用的访问结构是可能的。从现在开始,除非
特别强调,访问结构是指单调访问结构。\par
%%%%%%%%%%%%%%%%%%%%%%%%%%%%%%%%
\subsection{CP-ABE方案}
一个CP-ABE机制包含四个基本的算法:Setup,Encrypt,KeyGen和Decrypt。\\
\textbf{Setup}\quad Setup算法的输入只有隐含的安全参数,输出公共参数PK和主密钥MK。\\
\textbf{Encrypt(Pk,$M$,$\mathbb{A}$)}\quad 加密算法的输入是公共参数PK,明文$M$和属性全集
上的访问结构$\mathbb{A}$。算法加密明文$M$,产生密文CT,只有用户持有满足访问结构的属性
集合才能解密消息。我们假定密文隐含访问结构$\mathbb{A}$。\\
\textbf{Key Generation(MK,$S$)}\quad
密钥生成算法的输入是主密钥MK和描述私钥的属性集合$S$,输出是私钥SK。\\
\textbf{Decrypt(PK,CT,SK)}\quad
解密算法的输入是公共参数PK,包含访问策略$\mathbb{A}$的密文CT和属性集合$S$生成的私钥SK。
如果集合$S$满足访问结构$\mathbb{A}$那么算法会解密密文CT返回消息$M$。\par
%%%%%%%%%%%%%%%%%%%%%%%%%%%%%%%%
现在我们将描述CP-ABE方案的安全模型。如同IBE方案\cite{Shamir:IBC,BF:IBE,Cocks:IBE}的安全
模型,CP-ABE方案的安全模型允许敌手询问任何比能解密密文的私钥,即在我们的安全定义中敌手
挑战访问结构$\mathbb{A}^{*}$的解密,并且可以询问不满足$\mathbb{S}^{*}$的私钥$S$。
%%%%%%%%%%%%%%%%%%%%%%%%%%%%%%%%
\subsection{CP-ABE安全模型}
现在我们给出正式的安全游戏。\\
\textbf{Setup}\quad 挑战者运行Setup算法并将得到的公共参数PK交给敌手。\\
\textbf{阶段1}\quad 敌手反复生成对应于属性集$S_1,...,S_{q_1}$的私钥。\\
\textbf{Challenge}\quad 敌手提供两个等长的信息$M_0$和$M_1$。除此之外,敌手提供一个阶段1
中所有属性集合$S_1,...,S_{q1}$都不满足的挑战访问结构$\mathbb{A^*}$。挑战者随机选取
$b\in{0,1}$,在访问结构$\mathbb{A^*}$下加密$M_b$,并将密文$CT^*$敌手。\\
\textbf{阶段2}\quad 在属性集合$S_{q_1},...,S_q$不能满足对应的挑战访问结构的限制下重复
阶段1。\\
\textbf{Guess}\quad 敌手输出对$b$的猜测$b'$。\par
%%%%%%%%%%%%%%%%%%%%%%%%%%%%%%%%
\vspace{3mm}
在上述游戏中敌手$\mathcal{A}$的优势被定义为$\mathrm{Pr}[b'=b]-\frac{1}{2}$。我们注意到这个模型可以通过在阶段1和阶段2中允许解密敌手询问的明文来扩展为可以处理选择明文攻击的情形。
\begin{Definition}
在上述游戏中,如果多项式时间内敌手最多有可以忽略的优势,那么CP-ABE方案是安全的。
\end{Definition}
%%%%%%%%%%%%%%%%%%%%%%%%%%%%%%%%%%%%%%%%%%%%%%%%%%%%%%%
%%%%%%%%%%%%%%%%%%%%%%%%%%%%%%%%%%%%%%%%%%%%%%%%%%%%%%%
\section{双线性映射}
我们提出一些与群相关的高效可计算双线型映射的事实。\par
设$\mathbb{G}_0$和$\mathbb{G}_1$是阶为素数$p$的两个乘法循环群,$g$是$\mathbb{G}_0$
的生成元,$e:\mathbb{G}_0\times \mathbb{G}_0\to \mathbb{G}_1$。双线性映射$e$具有以
下性质:
\begin{enumerate}
\setlength{\itemsep}{0pt}
\item 双线性:对于任意$u,v\in \mathbb{G}_0$和任意$a,b\in \mathbb{Z}_p$,我们有
$e(u^a,v^b) =e(u,v)^{ab}$
\item 非退化性:$e(g,g)\neq 1$
\end{enumerate}
\par
如果$\mathbb{G}_0$中的群操作和双线性映射
$e:\mathbb{G}_0\times \mathbb{G}_0\to\mathbb{G}_1$都是高效可计算的,我们称
$\mathbb{G}_0$为双线性群。请注意由于$e(g^a,g^b)=e(g,g)^{ab}=e(g^b,g^a)$,
双线性映射$e$具有对称性。
